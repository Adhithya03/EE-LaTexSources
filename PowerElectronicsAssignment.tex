\documentclass[12pt]{article}
    \usepackage[dvipsnames]{xcolor}
    \usepackage{graphicx}
    \usepackage{tikz}
    \usepackage[margin=0.7in]{geometry}
    \usepackage{fancyhdr}
    \usepackage{xcolor}
    \usepackage{colortbl}
    \usepackage{hyperref}
    \usepackage{color,soul}


    
    \graphicspath{ {./images/} }
    \setlength{\parindent}{0pt}
    \setlength{\parskip}{5pt plus 1pt}
    \setlength{\headheight}{13.6pt}
    \newcommand\question[1]{\item\vspace{.25in}\hrule\textbf{#1}\vspace{.5em}\hrule\vspace{.10in}}
    \newcommand\answer{\vspace{.10in}\textbf{A: }}
    \pagestyle{fancyplain}
    \lhead{\textbf{\NAME}}
    \chead{\textbf{\HWNUM}}
    \rhead{\today}

    % https://coolors.co/palette/fdc921-fdd85d-fffdf7-99d6ea-6798c0
    \definecolor{urlBlue}{HTML}{6798C0}
    \definecolor{voxYellow}{HTML}{FDD85D}
    \definecolor{pigRed}{HTML}{bf211e}
    \setul{0.4ex}{0.3ex}
    \setulcolor{voxYellow}


    \newcommand\justurl[1]{\textcolor{urlBlue}{{\url{#1}}}}
    \newcommand\Red[2]{\textit{Answer from} \textbf{{#1}}\\ Page: {$#2$}}
    \renewcommand\u{\ul}
    \newcommand\p{\item}
    \newcommand\see[1]{\vspace{0.1in}{\small{\textit{Source: }\textcolor{urlBlue}{{\url{#1}}}}}}
    \newcommand\h[1]{\colorbox{voxYellow}{#1}}

    \newcommand\Dline{\begin{tikzpicture}\draw [thin,dash dot] (0,1) -- (16.5,1);\end{tikzpicture}}
    \newcommand\NAME{Adhithya}
    \newcommand\HWNUM{Power electronics U1 Q\&A}
    \setlength{\headheight}{20pt}


    \begin{document}\raggedright

\begin{itemize} 
    \p[$\rightarrow$] Material prepared with \LaTeX{} source code available at \justurl{https://github.com/Adhithya03/EE-LaTexSources/}\\

    \p[$\rightarrow$] Join our discord server for more resources, \justurl{https://discord.gg/aYzXDzPgzG}
\end{itemize}




\begin{enumerate}
    \question{What do you mean by thermal run away?}

    \answer With the increase in collector current $I_c$, collector power dissipation \u{increases} which \u{raises} the junction temperature that leads to further increase in collector current $I_c$. The process is cumulative and may lead to the eventual destruction of transistor. This phenomenon is known as \u{Thermal runaway}.

    \see{https://qr.ae/pvbjMP}

    
    \question{What is the difference between triggering and driving?}
    
    \u{Driving:} Applying power \u{continuously} to the gate or base of the power semiconductor device.
    \vspace{0.1in}

    \u{Triggering:} Applying a pulse at start and \u{removing it}.


    \question{Explain turn-on process of SCR using two transistor analogy.}
    

    \question{What are the advantages of UJT triggering over other methods?}
    
    \begin{enumerate}
        \p Pulse transformer in the circuit provides isolation between triggering circuit and power circuit.
        \p Has Switching time is in the range of nanoseconds.
    \end{enumerate}
 
    \see{https://web.archive.org/web/20220830091807/https://www.ktunotes.in/wp-content/uploads/2018/05/PE-M2-Ktunotes.in_.pdf}

    \question{What do mean by voltage commutation?}
    \answer A voltage source is impressed across the SCR to be turned off, mostly by an auxiliary SCR.
    
    \see{https://web.archive.org/web/20220830085415/https://guillaumeboivin.com/what-is-meant-by-voltage-commutation.html}
    


    \question{What do mean by current commutation?}
    
    \answer A current pulse is made to flow in the \u{reverse direction} through the conducting thyristor.


    \see{https://engineerscommunity.com/t/what-is-meant-by-current-commutation/1297}

    \question{What are the conditions for successful turn-off of SCR ?}
    \begin{enumerate}
    
    \p To turn off an SCR, it is required that its anode current should fall below the \u{holding current}, and, 

    \p \u{A reverse voltage} should be applied across the SCR for the sufficient time so that it goes to forward blocking mode from forward conduction mode.

    \end{enumerate}

    \see{https://web.archive.org/web/20220830092656/https://electricalbaba.com/scr-commutation-types/}

    \question{What are the advantages of GTO thyristor?}

    \begin{enumerate}
        \p No need of separate commutation circuit, Which reduces cost.
        
        \p Improved efficiency.

        \p Reduced acoustic and electromagnetic noise due to absence of commutation chokes.

        \p Faster turn-off, which increases the upper limiting frequency of the circuit.
    \end{enumerate}

    \see{M.H.Rashid 4th ed. Pg 481}

    \question{What do you mean by line commutation?}
        
    Line commutation is a Class-F SCR commutation technique in which, a thyristor is turned off due to \u{natural current zero} and voltage reversal after every half cycle.

    \question{List various forced commutation methods.}

    \begin{enumerate}
        \p Class-A Commutation (Load Commutation)
        \p Class-B Commutation (Resonant Pulse Commutation)
        \p Class-C Commutation (Complimentary Commutation)
        \p Class-D Commutation (Impulse Commutation)
        \p Class-E Commutation (External Pulse Commutation)
    \end{enumerate}


    \see{https://web.archive.org/web/20220830092656/https://electricalbaba.com/scr-commutation-types/}

    \question{Draw the pulse amplifier circuit used for triggering the SCR.}
    \includegraphics[width=6in]{pulseAmplifier.png}
    
    \question{Draw the driver circuit used for driving a MOSFET and BJT.}
    \begin{center}
        
    check sir's ppt.

    \end{center}

    \question{What do you mean by intelligent power module?}
    
    \answer ``Power module'' refers to the presence of a \u{power switching component} (usually an IGBT), and the module is ``intelligent'' because it includes additional control and \u{protection} circuitry.

    \see{https://web.archive.org/web/20220830094717/https://eepower.com/technical-articles/intelligent-power-modules-ipms-concepts-features-and-applications/}

    \question{What do you mean by secondary breakdown in BJT?}

     In a power transistor with a large junction area, under certain conditions of current and voltage, the current concentrates in a small spot of the base-emitter junction. This causes local heating, progressing into a short between collector and emitter.\\
     \see{https://web.archive.org/web/20220830095024/https://en.m.wikipedia.org/wiki/Safe_operating_area}
    
    \question{How do you design charging resistance of UJT firing circuit?}
    $$R_3 = {T\over C\times ln \left({1\over 1-\eta}\right)}$$
    \question{What do you mean by ringing circuit?}
    
    A circuit which has a capacitance in \u{parallel} with a resistance and inductance.

\end{enumerate}
\end{document}
